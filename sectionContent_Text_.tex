\section{Introduction}
%In organisms with adaptive immunity, cells expose fragments of proteins extracted from the cytosol. These peptide fragments, typically but not always 9 amino acids in length, are monitored by cytotoxic T cells, which recognize and kill infected or cancerous cells based on their viral, bacterial, mutant, or unusual peptides~\cite{Anderson_2004}. Presented peptides must bind to a major histocompatability complex I (MHC I) protein, which forms a platform for interaction with T cells. 

In most vertebrates, cytotoxic T-cells enforce multi-cellular order by killing both infected and cancerous cells. Each individual organism possesses a poly-clonal army of T-cells which collectively are able to distinguish unhealthy cells from healthy ones. This amazing feat is achieved through the winnowing and expansion of clonal T-cell populations possessing highly specific T-cell receptors (TCRs)~\cite{Blackman_1990}. Each distinct TCR recognizes a small number of similar peptides bound to an MHC molecule on the surface of a cell~\cite{Huseby_2005}. Though there are many steps in ``antigen processing''~\cite{Cresswell_2005}, it has become apparent that MHC binding is the most restrictive step. Peptide-MHC affinity prediction is the well-studied problem of predicting the binding strength of a given peptide and MHC pair~\cite{Lundegaard_2007}. Early approaches focused on ``sequence motifs''\cite{Sette_1989}, followed by regularized linear models, linear models with interaction terms such as SMM~\cite{Peters_2003}, and more recently the NetMHC family of predictors, a collection of related models based on ensembles of neural networks. Two of these predictors, NetMHC~\cite{Lundegaard_2008} and NetMHCpan~\cite{Nielsen_2007}, have emerged as the methods of choice across multiple fields of study within immunology, including virology~\cite{Lund_2011}, tumor immunology~\cite{Gubin_2015}, and autoimmunity~\cite{Abreu_2012}. 

% Initial approaches to predicting MHC ligands focused on ``sequence motifs''\cite{Sette_1989}, which were quickly replaced by a variety of regularized linear models, which themselves are consistently outperformed by regularized linear models with interaction terms such as SMM~\cite{Peters_2003}. The march toward black box non-linear models reached its local maximum with the NetMHC family of predictors, which are a collection of related models that utilize ensembles of neural networks. T

% Though there are many steps in ``antigen processing''~\cite{Cresswell_2005}, it has become apparent that MHC binding is the most restrictive step and consequently the most important sub-problem of predicting T-cell epitopes.

% (original): In most vertebrates, cytotoxic T-cells enforce multi-cellular order by killing both infected and cancerous cells. Each individual organism possesses a poly-clonal army of T-cells which collectively are able to distinguish rare unhealthy cells from healthy ones. This amazing feat is achieved through the winnowing and expansion of clonal T-cell populations possessing highly specific T-cell receptors (TCRs)~\cite{Blackman_1990}. Each distinct TCR is able to recognize a small number of similar peptides bound to an MHC molecule on the surface of a cell~\cite{Huseby_2005}. Though there are many steps in ``antigen processing''~\cite{Cresswell_2005} (the process by which protein fragments find themselves loaded onto membrane-bound MHCs), it has become apparent that MHC binding is the most restrictive step and consequently the most important sub-problem of predicting T-cell epitopes. 

NetMHC is an {\it allele-specific} method which trains a separate predictor for each allele's binding dataset, whereas NetMHCpan is a {\it pan-allele} method whose inputs are vector encodings of both a peptide and a subsequence of a particular MHC molecule. The conventional wisdom is that NetMHC performs better on alleles with many assayed ligands whereas NetMHCpan is superior for less well-characterized alleles~\cite{Gfeller_2016}.

In this paper we explore the space between {\it allele-specific} and {\it pan-allele} prediction by imputing the unobserved values of peptide-MHC affinities for which we have no measurements and using these imputed values for pre-training of allele-specific binding predictors.

%Paper summary:
%\begin{enumerate}
%\item Use cross-validation over the training data to choose the best ``large'' network configuration for alleles with many samples, as well as a ``smaller'' network which performs best on alleles with few samples.
%\item Use cross-validation over the training data to choose a most accurate imputation/matrix completion algorithm for 
%synthesizing pMHC binding affinities.
%\item Compare performance of the the best ``small'' and ``large'' networks (with and without pre-training on the synthesized samples) on an unseen test set vs. NetMHC, NetMHCpan, and SMM. 
%\end{enumerate}