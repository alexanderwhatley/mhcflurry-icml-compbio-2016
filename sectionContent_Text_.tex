\section{Introduction}
In most vertebrates, cytotoxic T-cells enforce multi-cellular order by killing both infected and cancerous cells~\cite{Anderson_2004}. Each individual organism possesses a poly-clonal army of T-cells which collectively are able to distinguish rare unhealthy cells from healthy ones. This amazing feat is achieved through the winnowing and expansion of clonal T-cell populations possessing highly specific T-cell receptors (TCRs)~\cite{Blackman_1990}. Each distinct TCR is able to recognize a small number of similar peptides bound to an MHC molecule on the surface of a cell~\cite{Huseby_2005}. Though there are many steps in ``antigen processing''~\cite{Cresswell_2005} (the process by which protein fragments find themselves loaded onto membrane-bound MHCs), it has become apparent that MHC binding is the most restrictive step and consequently the most important sub-problem of predicting T-cell epitopes. 

Two related algorithms - NetMHC and NetMHCpan - have emerged as the preferred methods for computational prediction of T-cell epitopes across several areas of immunology, including virology~\cite{Lund_2011}, tumor immunology~\cite{Gubin_2015}, and autoimmunity~\cite{Abreu_2012}. These algorithms differ primarily in that NetMHC is an `allele-specific'' method which trains a separate predictor for each allele's binding dataset. In cases where insufficient assay has been gathered for an allele, a NetMHC predictor cannot be trained. NetMHCpan, on the other hand, is a ``pan-allele'' method whose inputs are vector encodings of both the peptide and a subset of MHC molecule's primary sequence. The conventional wisdom is that NetMHC performs better on alleles with thousands of samples, whereas NetMHCpan is superior for alleles with only hundreds of samples or fewer. 


