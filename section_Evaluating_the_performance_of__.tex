\section{Evaluating the performance of a binding predictor}

Two datasets were used from a recent paper studying the relationship between training data and pMHC predictor accuracy\cite{Kim_2014}. The training dataset (BD2009) contained entries from IEDB\cite{Salimi_2012} up to 2009 and the test dataset (BLIND) contained IEDB entries from between 2010 and 2013 which did not overlap with BD2009 (Table~\ref{tab:datasets}).

\begin{table}[h!]
\centering
\begin{tabular}{l||cccc}
\toprule
{} & Alleles &  IC50 Measurements & Expanded 9mers \\
\midrule
BD2009 &     106 &                           137,654 &        470,170 \\
BLIND  &      53 &                           27,680 &         83,752 \\
\bottomrule
\end{tabular}
\caption{Train (BD2009) and test (BLIND) dataset sizes.}
\label{tab:datasets}
\end{table}

Throughout this paper we will evaluate a pMHC binding predictor using three different metrics:

\begin{itemize}
\item {\bf F$_1$ score}: Measures trade-off between sensitivity and specificity for predicting ``strong binders'' with affinities $<= 500$nM. 
\item {\bf AUC}: Area under the ROC curve. Estimates the probability that a ``strong binder'' peptide will be given a stronger predicted affinity than one whose ground truth affinity is $>500$nM. 
\item {\bf Kendall's $\tau$}: Rank correlation across the full spectrum of binding affinities.
\end{itemize}

